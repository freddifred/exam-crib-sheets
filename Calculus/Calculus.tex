% Calculus Examination Crib Sheet
% INCOMPLETE: SEE TODO MARKERS
%
% OWD 2023

% Examination Crib Sheet Common Preamble
% OWD 2023

\documentclass{article}
\usepackage[a4paper, margin=2em, landscape]{geometry}
\usepackage{array, xcolor, fancyhdr}
\usepackage[en-GB]{datetime2}
\usepackage[
        colorlinks = true,
        allcolors = darkgray
]{hyperref}

\title{\modulename\ Examination Crib Sheet (2022/23)}
\author{Oliver Dixon}

\setlength{\parindent}{0pt}

% Header height set-up, in accordance with the margins
\renewcommand{\headrulewidth}{0pt}
\setlength{\headheight}{18pt}
\setlength{\headsep}{1em}
\addtolength{\topmargin}{\headheight}
\makeatletter
\addtolength{\topmargin}{\Gm@tmargin}
\makeatother

% Footer height set-up
\renewcommand{\footrulewidth}{0pt}
\addtolength{\textheight}{-6em}

% Header and footer body set-up
\makeatletter
\fancypagestyle{fancy}{%
	\fancyhf{}
        \fancyhead[L]{\Large\textbf{\@title}}
        \fancyhead[R]{\Large Compiled by \@author\ on \today}
        \fancyfoot[L]{\Large\textsc{Side \thepage}}
        \fancyfoot[R]{\Large\url{https://github.com/oliverdixon/%
                exam-crib-sheets/\modulename}}
}
\makeatother
\pagestyle{fancy}


\newcommand{\modulename}{Calculus}

\begin{document}
%
% SIDE 1
%
\centering
\begin{tabular}{|m{.31\linewidth}|m{.31\linewidth}|m{.31\linewidth}|}
        \hline
        %
        \textbf{L'H{\^ o}pital's Rule}: If $f$ and $g$ are differentiable
        functions at $x_0$, $f(x_0)=g_(x_0)=0$, and $g^\prime(x_0)\neq 0$, then
        $\lim_{x\to x_0} f(x)/g(x)=\lim_{x\to x_0} f^\prime(x)/g^\prime(x)$. &
        %
        \textbf{The IVT}: Suppose $a<b$ and $f$ is continuous on $[a,b]$.
        Then, for every $y$ such that $\min(f(a),f(b)) < y < \max(f(a),f(b))$,
        there exist $x_0\in (a,b)$ s.t.\ $f(x_0)=y$. &
        %
        \textbf{The Chain Rule}: If $g$ is differentiable at $x$ and $f$ is
        differentiable at $g(x)$, then $f\circ g$ is differentiable at $x$, and
        $(f\circ g)^\prime(x)=f^\prime(g(x))g^\prime(x)$. \\
        %
        \hline
        %
        \textbf{The IFT}: If $f:I\to\mathbb{R}$ is continuous and strictly
        monotonic, then $f^{-1}:J\to I$ is also continuous, where $J=f(I)$ and
        $f^{-1}(f(x))=x$ and $f(f^{-1}(y))=y$. &
        %
        \textbf{The MVT}: If $f:[a,b]\to\mathbb{R}$ is continuous and
        differentiable on $(a,b)$, then there exist $x_0\in (a,b)$ such that
        $f^\prime(x_0)=\left[f(b)-f(a)\right]/(b-a)$. &
        %
        \textbf{Classifying CPs}: If $f:[a,b]\to\mathbb{R}$, $f^\prime$,
        $f^{\prime\prime}$ are sensibly defined, and $x_0\in (a,b)$ s.t.\ %
        $f^\prime(x_0)=0$, then $f^{\prime\prime}(x_0)>0$ means local min., and
        $f^{\prime\prime}(x_0)<0$ means local max. \\
        %
        \hline
        %
        \textbf{Taylor's Theorem (1)}: If $f\in C^{N+1}(I)$ and $x\in I$, then
        $f(x)=\sum_{n=0}^N \left[f^{(n)}(x_0)(x-x_0)^n\right]/n!+1/N!
        \int_{x_0}^x (x-t)^N f^{(N+1)}(t)\,\mathconst{d}t$. &
        %
        \textbf{Taylor's Theorem (2)}: The terms under the summation are the
        \emph{Taylor polynomial} of $f$ at $x_0$, of order $N$. The integral
        term is known as the \emph{error in integral form}. &
        %
        \textbf{Taylor's Theorem (3)}: The \emph{Lagrange form} of the error is
        $R_N(x)=\left[(x-x_0)^{N+1}f^{(N+1)(c)}\right]/(N+1)!$, for some $c$
        between $x_0$ and $x$. \\
        %
        \hline
        %
        \textbf{Diff. Eq. (1)}: If $u^\prime(x)=cu(x)$, where
        $c\in\mathbb{R}\setminus\{0\}$ and $A$ is an arbitrary constant, then
        $u(x)=A\mathconst{e}^{cx}$. &
        %
        \textbf{Diff. Eq. (2)}: If $u^{\prime\prime}(x)=-c^2u(x)$, then
        $A\cos(cx)+B\sin(cx)$, where $A$ and $B$ are arbitrary constants. &
        %
        \textbf{Diff. Eq. (3)}: If $u^{\prime\prime}(x)=c^2u(x)$, then
        $u(x)=A\mathconst{e}^{cx}+B\mathconst{e}^{-cx}=C\cosh(cx)+D\sinh(cx)$,
        for arb.\ constants $C, D$. \\
        \hline
        %
        \textbf{Simple Diff. Eqs.}: A \emph{simple differential equation} has
        the form $y^\prime(x)=f(x)$, and has solutions $y=\int f(x)\,
        \mathconst{d}x+C$, for some arbitrary constant $C$. &
        %
        \textbf{Separable Diff. Eqs.}: A \emph{separable differential equation}
        has the form $y^\prime(x)=f(x)/g(y)$. It has solutions $G(y)=F(x)+C$,
        where $F^\prime=f$ and $G^\prime=g$. &
        %
        \textbf{Integrating Factors (1)}: A first-order ODE is \emph{linear} if
        it has the form $a(x)y^\prime(x)+b(x)y+c(x)=0$. In \emph{standard form},
        this is $y^\prime(x)=P(x)y+Q(x)=0$\ \ldots \\
        \hline
        %
        \textbf{Integrating Factors (2)}: \ldots\ This can be solved to give
        $y=\left[\int Q(x)F(x)\,\mathconst{d}x+C\right]/F(x)$, where
        $F(x)=\exp\int P(x)\,\mathconst{d}x$ is the \emph{integrating factor}. &
        %
        \textbf{Derivative of Arc Sine}: \smash{$\dfrac{\mathconst{d}}%
        {\mathconst{d}x}\arcsin(x)=\dfrac{1}{\sqrt{1-x^2}}$} &
        %
        \textbf{Derivative of Arc Cosine}: \smash{$\dfrac{\mathconst{d}}%
        {\mathconst{d}x}\arccos(x)=\dfrac{-1}{\sqrt{1-x^2}}$} \\
        %
        \hline
        %
        \textbf{Derivative of Arc Tangent}: \smash{$\dfrac{\mathconst{d}}%
        {\mathconst{d}x}\arctan(x)=\dfrac{1}{1+x^2}$} &
        %
        \textbf{Radian Measure (1)}: If $(x,y)\in\mathbb{R}^2$ with
        $(x,y)\neq(0,0)$, then there is a unique solution to $x=r\cos\theta$
        and $y=r\sin\theta$ for $\theta\in(-\pi,\pi]$ and $r>0$. &
        %
        \textbf{Radian Measure (2)}: If $x>0$, then $\theta=\arctan(y/x)$. If
        $x=0$, $\theta=\sgn(y)\pi/2$. If $x<0$, then
        $\theta=\arctan(y/x) + \pi$ if $y\geq 0$, or $\theta=\arctan(y/x)-\pi$
        otherwise. \\
        %
        \hline
        %
        \textbf{Complex Circular Trigonometric Functions}: For
        $z\in\mathbb{C}$, $\sin(z)=(\mathconst{e}^{\mathconst{i}z}-
        \mathconst{e}^{-\mathconst{i}z})/2$, and $\sin(z)=(\mathconst{e}%
        ^{\mathconst{i}z}+\mathconst{e}^{-\mathconst{i}z})/2$. Therefore,
        $\tan(z)=\mathconst{i}(\mathconst{e}^{-\mathconst{i}z-\mathconst{e}%
        ^{\mathconst{i}z}})/(\mathconst{e}^{-\mathconst{i}z}+\mathconst{e}%
        ^{\mathconst{i}z})$. &
        %
        \textbf{Complex Hyperbolic Trigonometric Functions}: For
        $z\in\mathbb{C}$, $\sinh(z)=(\mathconst{e}^z-\mathconst{e}^{-z})/2$,
        $\cosh(z)=(\mathconst{e}^z+\mathconst{e}^{-z})/2$, and
        $\tanh(z)=\sinh(z)/\cosh(z)$. &
        %
        \textbf{Trigonometric Identities (Hyperbolic Form)}: For $x$ and $y$,
        $\sinh(x+y)=\sinh(x)\cosh(y) + \cosh(x)\sinh(y)$, and
        $\cosh(x+y)=\cosh(x)\cosh(y) + \sinh(x)\sinh(y)$. \\
        %
        \hline
\end{tabular}
\clearpage
%
% SIDE 2
%
\begin{tabular}{|m{.31\linewidth}|m{.31\linewidth}|m{.31\linewidth}|}
        \hline
        1 & 2 & 3 \\
        \hline
        4 & 5 & 6 \\
        \hline
\end{tabular}
\end{document}

