% Calculus Examination Crib Sheet
% INCOMPLETE: SEE TODO MARKERS
%
% OWD 2023

% Examination Crib Sheet Common Preamble
% OWD 2023

\documentclass{article}
\usepackage[a4paper, margin=2em, landscape]{geometry}
\usepackage{array, xcolor, fancyhdr}
\usepackage[en-GB]{datetime2}
\usepackage[
        colorlinks = true,
        allcolors = darkgray
]{hyperref}

\title{\modulename\ Examination Crib Sheet (2022/23)}
\author{Oliver Dixon}

\setlength{\parindent}{0pt}

% Header height set-up, in accordance with the margins
\renewcommand{\headrulewidth}{0pt}
\setlength{\headheight}{18pt}
\setlength{\headsep}{1em}
\addtolength{\topmargin}{\headheight}
\makeatletter
\addtolength{\topmargin}{\Gm@tmargin}
\makeatother

% Footer height set-up
\renewcommand{\footrulewidth}{0pt}
\addtolength{\textheight}{-6em}

% Header and footer body set-up
\makeatletter
\fancypagestyle{fancy}{%
	\fancyhf{}
        \fancyhead[L]{\Large\textbf{\@title}}
        \fancyhead[R]{\Large Compiled by \@author\ on \today}
        \fancyfoot[L]{\Large\textsc{Side \thepage}}
        \fancyfoot[R]{\Large\url{https://github.com/oliverdixon/%
                exam-crib-sheets/\modulename}}
}
\makeatother
\pagestyle{fancy}


\newcommand{\modulename}{Calculus}

\begin{document}
%
% SIDE 1
%
\centering
\begin{tabular}{|m{.31\linewidth}|m{.31\linewidth}|m{.31\linewidth}|}
\hline
%
\textbf{L'H{\^ o}pital's Rule}: If $f$ and $g$ are differentiable functions at
$x_0$, $f(x_0)=g_(x_0)=0$, and $g^\prime(x_0)\neq 0$, then $\lim_{x\to x_0}
f(x)/g(x)=\lim_{x\to x_0} f^\prime(x)/g^\prime(x)$. &
%
\textbf{The IVT}: Suppose $a<b$ and $f$ is continuous on $[a,b]$.  Then, for
every $y$ such that $\min(f(a),f(b)) < y < \max(f(a),f(b))$, there exist $x_0\in
(a,b)$ s.t.\ $f(x_0)=y$. &
%
\textbf{The Chain Rule}: If $g$ is differentiable at $x$ and $f$ is
differentiable at $g(x)$, then $f\circ g$ is differentiable at $x$, and $(f\circ
g)^\prime(x)=f^\prime(g(x))g^\prime(x)$. \\
%
\hline
%
\textbf{The IFT}: If $f:I\to\mathbb{R}$ is continuous and strictly monotonic,
then $f^{-1}:J\to I$ is also continuous, where $J=f(I)$ and $f^{-1}(f(x))=x$ and
$f(f^{-1}(y))=y$. &
%
\textbf{The MVT}: If $f:[a,b]\to\mathbb{R}$ is continuous and differentiable on
$(a,b)$, then there exist $x_0\in (a,b)$ such that
$f^\prime(x_0)=\left[f(b)-f(a)\right]/(b-a)$. &
%
\textbf{Classifying CPs}: If $f:[a,b]\to\mathbb{R}$, $f^\prime$,
$f^{\prime\prime}$ are sensibly defined, and $x_0\in (a,b)$ s.t.\ %
$f^\prime(x_0)=0$, then $f^{\prime\prime}(x_0)>0$ means local min., and
$f^{\prime\prime}(x_0)<0$ means local max. \\
%
\hline
%
\textbf{Taylor's Theorem (1)}: If $f\in C^{N+1}(I)$ and $x\in I$, then
$f(x)=\sum_{n=0}^N \left[f^{(n)}(x_0)(x-x_0)^n\right]/n!+1/N!  \int_{x_0}^x
(x-t)^N f^{(N+1)}(t)\,\mathconst{d}t$. &
%
\textbf{Taylor's Theorem (2)}: The terms under the summation are the
\emph{Taylor polynomial} of $f$ at $x_0$, of order $N$. The integral term is
known as the \emph{error in integral form}. &
%
\textbf{Taylor's Theorem (3)}: The \emph{Lagrange form} of the error is
$R_N(x)=\left[(x-x_0)^{N+1}f^{(N+1)(c)}\right]/(N+1)!$, for some $c$ between
$x_0$ and $x$. \\
%
\hline
%
\textbf{Diff. Eq. (1)}: If $u^\prime(x)=cu(x)$, where
$c\in\mathbb{R}\setminus\{0\}$ and $A$ is an arbitrary constant, then
$u(x)=A\mathconst{e}^{cx}$. &
%
\textbf{Diff. Eq. (2)}: If $u^{\prime\prime}(x)=-c^2u(x)$, then
$A\cos(cx)+B\sin(cx)$, where $A$ and $B$ are arbitrary constants. &
%
\textbf{Diff. Eq. (3)}: If $u^{\prime\prime}(x)=c^2u(x)$, then
$u(x)=A\mathconst{e}^{cx}+B\mathconst{e}^{-cx}=C\cosh(cx)+D\sinh(cx)$, for arb.\
constants $C, D$. \\
\hline
%
\textbf{Simple Diff. Eqs.}: A \emph{simple differential equation} has the form
$y^\prime(x)=f(x)$, and has solutions $y=\int f(x)\, \mathconst{d}x+C$, for some
arbitrary constant $C$. &
%
\textbf{Separable Diff. Eqs.}: A \emph{separable differential equation} has the
form $y^\prime(x)=f(x)/g(y)$. It has solutions $G(y)=F(x)+C$, where $F^\prime=f$
and $G^\prime=g$. &
%
\textbf{Integrating Factors (1)}: A first-order ODE is \emph{linear} if it has
the form $a(x)y^\prime(x)+b(x)y+c(x)=0$. In \emph{standard form}, this is
$y^\prime(x)=P(x)y+Q(x)=0$\ \ldots \\
\hline
%
\textbf{Integrating Factors (2)}: \ldots\ This can be solved to give
$y=\left[\int Q(x)F(x)\,\mathconst{d}x+C\right]/F(x)$, where $F(x)=\exp\int
P(x)\,\mathconst{d}x$ is the \emph{integrating factor}. &
%
\textbf{Derivative of Arc Sine}: \smash{$\dfrac{\mathconst{d}}%
{\mathconst{d}x}\arcsin(x)=\dfrac{1}{\sqrt{1-x^2}}$} &
%
\textbf{Derivative of Arc Cosine}: \smash{$\dfrac{\mathconst{d}}%
{\mathconst{d}x}\arccos(x)=\dfrac{-1}{\sqrt{1-x^2}}$} \\
%
\hline
%
\textbf{Derivative of Arc Tangent}: \smash{$\dfrac{\mathconst{d}}%
{\mathconst{d}x}\arctan(x)=\dfrac{1}{1+x^2}$} &
%
\textbf{Radian Measure (1)}: If $(x,y)\in\mathbb{R}^2$ with $(x,y)\neq(0,0)$,
then there is a unique solution to $x=r\cos\theta$ and $y=r\sin\theta$ for
$\theta\in(-\pi,\pi]$ and $r>0$. &
%
\textbf{Radian Measure (2)}: If $x>0$, then $\theta=\arctan(y/x)$. If $x=0$,
$\theta=\sgn(y)\pi/2$. If $x<0$, then $\theta=\arctan(y/x) + \pi$ if $y\geq 0$,
or $\theta=\arctan(y/x)-\pi$ otherwise. \\
%
\hline
%
\textbf{Complex Circular Trigonometric Functions}: For $z\in\mathbb{C}$,
$\sin(z)=(\mathconst{e}^{\mathconst{i}z}- \mathconst{e}^{-\mathconst{i}z})/2$,
and $\sin(z)=(\mathconst{e}%
^{\mathconst{i}z}+\mathconst{e}^{-\mathconst{i}z})/2$. Therefore,
$\tan(z)=\mathconst{i}(\mathconst{e}^{-\mathconst{i}z}-\mathconst{e}%
^{\mathconst{i}z})/(\mathconst{e}^{-\mathconst{i}z}+\mathconst{e}%
^{\mathconst{i}z})$. &
%
\textbf{Complex Hyperbolic Trigonometric Functions}: For $z\in\mathbb{C}$,
$\sinh(z)=(\mathconst{e}^z-\mathconst{e}^{-z})/2$,
$\cosh(z)=(\mathconst{e}^z+\mathconst{e}^{-z})/2$, and
$\tanh(z)=\sinh(z)/\cosh(z)$. &
%
\textbf{Trigonometric Identities (Hyperbolic Form)}: For $x$ and $y$,
$\sinh(x+y)=\sinh(x)\cosh(y) + \cosh(x)\sinh(y)$, and
$\cosh(x+y)=\cosh(x)\cosh(y) + \sinh(x)\sinh(y)$. \\
%
\hline
\textbf{2nd-Order ODE (MD.1)}: Consider $R(y^{\prime\prime},y^\prime,x)=0$. To
solve for $y(x)$, we define a new dependent variable as the derivative of the
old dependent variable &
%
\textbf{\ldots\ (MD.2)}: We then solve the resulting first-order ODE, and
integrate the solution. This works the cases of \emph{missing dependent
variables}. &
%
\textbf{2nd-Order ODE (MI.1)}: Consider $R(y^{\prime\prime},y^\prime,y)=0$. To
solve this \emph{autonomous ODE}, we first define a new independent variable as
the old dependent variable. \\
%
\hline
%
\textbf{\ldots\ (MI.2)}: Define a dependent variable as the derivative of the
old dependent variable. Rewrite the expression in terms of these new variables,
and solve. &
%
\textbf{\ldots\ (MI.3)}: Rewrite the solution in terms of the original
variables, and solve the resulting first-order differential equation. &
%
\textbf{2nd-Order ODE (HC.1)}: Consider a homogeneous linear ODE in $y(x)$ with
constant coeffs. Take an ansatz of $\mathconst{e}^{\lambda x}$, substitute this
into the auxiliary equation, and solve. \\
%
\hline
%
\textbf{\ldots\ (HC.2)}: If $\lambda\in\mathbb{R}$ is a root of the aux.  eq.,
then $\mathconst{e}^{\lambda x}$ is a solution to the ODE.  If
$\alpha\pm\mathconst{i}\beta\in\mathbb{C}$ are roots, then
$\mathconst{e}^{\alpha x}\cos\beta x$ and $\mathconst{e}^{\alpha x}\sin\beta x$
are ODE solutions. &
%
\textbf{\ldots\ (HC.3)}: If $\lambda$ is an $m$-times repeated root of the aux.
eq., with $m\leq n$, then multiplying these solutions by powers of $x$, up to
$x^{m-1}$, gives more solutions. &
%
\textbf{\ldots\ (HC.4)}: The general solution of the ODE is an arbitrary linear
combination of these real and complex solutions. \\
%
\hline
%
\textbf{2nd-Order ODE (IL.1)}: For an \emph{inhomogeneous linear ODE}, first
find the general solution of the corresponding homogeneous equation. &
%
\textbf{\ldots\ (IL.2)}: Find one solution of the inhomogeneous ODE, and
sum it with the solution to the homogeneous ODE for the general solution. &
%
\textbf{2nd-Order ODE (IC.1)}: For an \emph{inhomogeneous linear ODE with
constant coeffs.}, take an ansatz which is of the same type as the RHS, with
undetermined coeffs. \\
%
\hline
%
\textbf{\ldots\ (IC.2)}: If this ansatz ``overlaps'' with the general
solution of the homogeneous ODE, multiply that part of the guess by $x$. &
%
\textbf{\ldots\ (IC.3)}: Insert this into the ODE and determine the coeffs.
Substitute these values into the ansatz, and take the sum of the particular and
general solution. &
%
\textbf{2nd-Order ODE (CP.1)}: For a system of coupled ODEs in $x(t)$ and
$y(t)$, find $\ddot{x}$ and $\dot{y}$. Use $\dot{y}$ to eliminate $\dot{y}$, and
use $\dot{x}$ to eliminate $y$ in the ODE. \\
%
\hline
\end{tabular}
\clearpage
%
% SIDE 2
%
\begin{tabular}{|m{.31\linewidth}|m{.31\linewidth}|m{.31\linewidth}|}
\hline
%
\textbf{\ldots\ (CP.2)}: Find the general solution of the resulting ODE for
$x(t)$, and compute $\dot{x}(t)$. Use $\dot{x}$ to write $y$ in $x$ and
$\dot{x}$, and compute $y(t)$ from the general solution of $x(t)$. &
%
\textbf{Basic FS}: The \emph{Fourier Series} for $f:[-\pi,\pi]\to \mathbb{R}$ is
$S(x)=a_0/2 + \sum_{n=1}^\infty (a_n\cos nx + b_n\sin nx)$, with
$a_n=\int_{-\pi}^\pi f(x)\cos nx\,\mathconst{d}x/\pi$ and $b_n=\int_{-\pi}^\pi
f(x)\sin nx\,\mathconst{d}x/\pi$. &
%
\textbf{FS Cautions}: Check for any symmetries of the function, or suggestions
of a constant equalling zero. For the cosine terms $a_n$, $n=0,1,2,\ldots$ For
sine terms $b_n$, $n=1,2,\ldots$ \\
%
\hline
%
\textbf{Periodic Extensions}: If $f:[-\pi,\pi)\to\mathbb{R}$, then its
\emph{periodic extension} $\tilde{f}:\mathbb{R}\to\mathbb{R}$ is defined by
$\tilde{f}(x+2\pi k)\coloneqq f(x)$ for $k\in\mathbb{Z}$ and $-\pi\leq x<\pi$. &
%
\textbf{FCT (1)}: If $f:[-\pi,\pi]\to\mathbb{R}$ is a PWCD${}^\dagger$ function,
and $\tilde{f}:\mathbb{R}\to\mathbb{R}$ is its $2\pi$-PE, then at
$x\in\mathbb{R}$, the FS of $f$ converges to
$\lim_{N\to\infty}S_N(x)=S(x)=[\tilde{f}(x^+)+\tilde{f}(x^-)]/2$. &
%
\textbf{FCT (2)}: If $\tilde{f}$ is continuous at $x$, then $S(x)=\tilde{f}(x)$.

[\hspace{1pt}${}^\dagger$ Piecewise continuously differentiable function.] \\
%
\hline
%
\textbf{PT}: If $f:[-\pi,\pi]\to\mathbb{R}$ is a PWCD with Fourier coefficients
$a_0$, $a_n$, and $b_n$ for $n\in\mathbb{N}$, then $\int_{-\pi}^\pi f^2(x)\,
\mathconst{d}x/\pi \equiv a_0^2/2+\sum_{n=1}^\infty (a_n^2+b_n^2)$. &
%
\textbf{Half-Range Series}: For $f:[0,\pi]\to\mathbb{R}$, $S_c(x)=a_0/2 +
\sum_{n=1}^\infty a_n\cos nx$, and $S_s(x)=\sum_{n=1}^\infty b_n\sin nx$, where
$a_n=2\int_0^\pi f(x)\cos nx\,\mathconst{d}x/\pi$ and
$b_n=2\int_0^\pi f(x)\sin nx\,\mathconst{d}x/\pi$. &
%
\textbf{Complex Exponential Series}: For complex-valued coefficients
$c_n\in\mathbb{C}$, $S(x)=\sum_{n=-\infty}^\infty
c_n\mathconst{e}^{\mathconst{i}nx}$, where $c_n=\int_{-\infty}^\infty
f(x)\mathconst{e}^{-\mathconst{i}nx}\,\mathconst{d}x /(2\pi)$ and
$\overline{c_n}=c_{-n}$ with $n\in\mathbb{N}\cup \{0\}$. \\
%
\hline
%
\textbf{FS on Other Intervals (1)}: For a function over $[-L/2,L/2]$,
$S(x)=a_0/2+\sum_{n=1}^\infty[a_n\cos(2n\pi x/L)+b_n\sin(2n\pi x/L)]$, where
\ldots&
%
\textbf{FS on Other Intervals (2)}: \ldots\ The cosine coefficients
$a_n=2\int_{-L/2}^{L/2} f(x)\cos(2n\pi x/L)\,\mathconst{d}x/L$, and
the sine coefficients
$b_n=2\int_{-L/2}^{L/2} f(x)\sin(2n\pi x/L)\,\mathconst{d}x/L$. &
%
{} \\
\hline
\end{tabular}
\end{document}

